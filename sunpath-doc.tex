\documentclass{article}
\usepackage{tikz}

\usetikzlibrary{calc,math}
\usetikzlibrary[through]


\begin{document}
\makeatletter
%% Radius of the 0-grad Altitude Circle of the sunpath diagram.
%% Its unit is default unit of PGF xy Coordinate
%% Default 5.5  (=5.5cm for nomal scale of an tikzpicture)
\pgfkeys{/tikz/.cd,
  spradius/.store in=\spradius,
  spradius=5.5,                  %% initial value, in PGF xy Coordinate
  altitude projection/.store in = \spprojection,
  altitude projection=spherical
}
% set azi (=azimuth) und alt (=altitude)
\tikzset{
  cs/azi/.store in=\tikz@cs@azi,
  cs/alt/.store in=\tikz@cs@alt,
  declare function = {
      spherical(\alt) = \spradius * cos(\alt);
      equidistance(\alt) = \spradius - \spradius*\alt/90;
      altradius(\alt) = \spprojection(\alt); 
      aziangle(\x) = 90 - \x;
  }
}

%% Define the new coordinate system sunpath
%% Usage example:
%% 
%% \coordinate (sunpath cs:azi=110,alt=65); 
%%
\tikzdeclarecoordinatesystem{sunpath}%
{
    \tikzset{cs/.cd,azi=0,alt=0,#1}
    \tikzmath{
      %\r = \spradius - \spradius*\tikz@cs@alt/90;
      \r = altradius(\tikz@cs@alt);
      \angle = aziangle(\tikz@cs@azi);
    }
    \pgfpointadd{\pgfpointxy{0}{0}}{% Origin
        %\pgfpointpolarxy{90-\tikz@cs@azi}{\r}
        \pgfpointpolarxy{\angle}{\r}
    }
}


%% Optical styles
\tikzset{
%% Grid of diagram (used for azimuth lines and altitude circles)
  sunpath grid/.style={help lines,color=blue!45!white!80},
%% Azimuth ticks around the 0-degree-altidude circle
  sunpath tick/.style={draw,thick,color=blue!90!white!80},
%% Azimuth minor ticks  
  sunpath minor tick/.style={draw,thin,color=blue!90!white!80},
%% Format for the label of the altitude values
  altitude label/.style={font=\footnotesize\sffamily,fill=white,minimum width={width("90")+2pt},inner sep=0.5pt},
  azimuth label/.style={font=\footnotesize\sffamily,minimum width={width("360")+2pt},inner sep=0.5pt},
}

%% Place 4 geographie direction on the diagram
\newcommand{\drawgeodirection}{
  \foreach \dname / \dgrad \ in {N/0, E/90, S/180, W/270}{
      \tikzmath{
        \polarangle = aziangle(\dgrad);
      }
      \coordinate (D) at (\polarangle:\spradius cm + 22pt);
      \node[anchor=270-\dgrad] at (D) {\dname};
  };
}

%% Draw altitude circles at given altitude, for example {0,10,...,80} or {0,15,...,90}
\newcommand{\drawaltitudecirle}[1]{
  \foreach \altitude in #1 {
    \coordinate (A) at (sunpath cs:azi=0,alt=\altitude) ;  
    \path[draw,sunpath grid] (0,0) circle[radius=altradius(\altitude)];
  }
}

%% Draw altitude value at given altitude, for example {0,10,...,80} or {0,15,...,90}
%% The altitude value are placed along the line from (0,0) to Nord
\newcommand{\drawaltitudelabel}[1]{
  \foreach \altitude in #1 {
      \coordinate (A) at (sunpath cs:azi=0,alt=\altitude) ;      
      \node [anchor=east,altitude label] at (A) {\altitude};  
    }
}

\newcommand{\drawazimuthlabel}[1]{
  \foreach \azimuth in #1 {
      \tikzmath{
        \polarangle = aziangle(\azimuth);
      }
      \coordinate (D) at (\polarangle:\spradius cm + 13pt);
      \node[azimuth label] at (D) {\azimuth};
  }
}


%% Draw azimuth lines from center of the diagram to the 0-grad altitude
\newcommand{\drawazimuthline}[3]{
  \foreach \azimuth in #1{
      \draw[sunpath grid] (sunpath cs:azi=\azimuth,alt={#2}) -- (sunpath cs:azi=\azimuth,alt={#3});
  }
}

%% Draw major and minor azimuth ticks around the 0-grad altitude circle
\newcommand{\drawazimuthtick}{
  \foreach \azimuth in {10,20,...,360}{
    \tikzmath{      
        \pa = aziangle(\azimuth);
    }
    \path[sunpath tick] (\pa:\spradius) -- (\pa:{\spradius cm+6pt});
  }
  
  \foreach \azimuth in {1,2,...,360}{
    \tikzmath{      
        \pa = aziangle(\azimuth);
    }
    \path[sunpath minor tick] (\pa:\spradius) -- (\pa:{\spradius cm+2.5pt});
  }
  
  \foreach \azimuth in {15,30,...,360}{
      \tikzmath{      
          \pa = aziangle(\azimuth);
      }
      \path[sunpath minor tick] (\pa:\spradius) -- (\pa:{\spradius cm+5pt});
    }  
}


\begin{tikzpicture}[spradius=6]%[x=5cm,y=5cm,spradius=1]
\drawaltitudecirle{{0,10,...,80,85}}
\drawazimuthline{{0,10,...,360}}{85}{70}
\drawazimuthline{{0,5,...,360}}{80}{0}

\drawazimuthtick

\drawgeodirection
\drawaltitudelabel{{10,20,...,80,85}}
\drawazimuthlabel{{0,15,...,350}} ;  
\end{tikzpicture}


\begin{tikzpicture}[spradius=6,altitude projection=equidistance]%[x=5cm,y=5cm,spradius=1]
\drawaltitudecirle{{0,10,...,80}}

\drawazimuthline{{0,10,...,360}}{80}{0}

\drawazimuthtick

\drawgeodirection
\drawaltitudelabel{{10,20,...,80}}
\drawazimuthlabel{{0,15,...,350}} ;  
\end{tikzpicture}
\end{document}














































