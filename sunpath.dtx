% \iffalse meta-comment
%
% File: sunpath.dtx Copyright (C) 2019 Hong-Phuc Bui
%
% It may be distributed and/or modified under the conditions of the
% LaTeX Project Public License (LPPL), either version 1.3c of this
% license or (at your option) any later version.  The latest version
% of this license is in the file
%
%    https://www.latex-project.org/lppl.txt
%
%
% -----------------------------------------------------------------------
%
% The development version of the bundle can be found at
%
%    https://github.com/hpb-htw/sunpath
%
% for those people who are interested.
%
% -----------------------------------------------------------------------
% \fi
%
% \iffalse
%<package>\NeedsTeXFormat{LaTeX2e}[2018/12/01]
%<package>\ProvidesPackage{sunpath}[2024/10/10 v0.1-Alpha Draw Sun Path]
% \fi
%
% \iffalse
%<*driver>
\documentclass[full]{l3doc}

\usepackage{fontspec}
\usepackage{luaotfload}
\usepackage{tikz}


\begin{document}
  \DocInput{\jobname.dtx}
  % ^^A%%%%%%%%%%%%%%%%%%%%%%%%%%%%%%%%%%%%%%%%%%%%%%%%%%%%%%%%%%
  \PrintChanges
  \PrintIndex
\end{document}
%</driver>
% \fi
%
%
%
% \title{^^A
%   \pkg{sunpath} -- Draw Sun Path^^A
%   \thanks{This file describes \fileversion, ^^A
%     last revised \filedate.}\\[1ex]^^A
%     \normalsize{Reference}^^A
% }^^A
%
% \author{^^A
%  Hồng-Phúc Bùi^^A
%  \thanks{^^A
%    E-mail:
%    \href{mailto:Hồng-Phúc Bùi}
%      {hong-phuc.bui (at) htwsaar dot de}^^A
%   }^^A
% }
%
% \date{Released \filedate}
%
% \parindent0pt
%
% \maketitle
% \tableofcontents
%
% \section{Hints}
% Usage of this Package can be found in \texttt{sunpath.usage.pdf}.
% This document show only generated reference of commands in this Package.
%
% ^^A %%%%%%%%%%%%%%%%%%%%%%%%%%%%%%%%%%%%%%%%%%%%%%%%%%%%%%%%%%%%%%%%%%%%%%%%%%%%%%%%%%%%%%
% \section{Implementation}
%
% \changes{v0.1-Alpha}
%         {2024/10/10}
%         {Initial implementation}
%
%
% \subsection{Package Dependenies}
%
%    \begin{macrocode}
\RequirePackage{expl3}
\RequirePackage{tikz}
%    \end{macrocode}
%
% Load necsessary \texttt{tikz}-libraries.
%    \begin{macrocode}
\usetikzlibrary{calc,math,through}
%    \end{macrocode}

%     Setup  options for tikzpicture environment
%
%
%
%
\pgfkeys{/tikz/.cd,
  spradius/.store in=\spradius,
  spradius=5.5,
  altitude projection/.store in = \spprojection,
  altitude projection=spherical
}

% Define component azi and alt for the coordinate system \texttt{sunpath}
\makeatletter
\tikzset{
  cs/azi/.store in=\tikz@cs@azi,
  cs/alt/.store in=\tikz@cs@alt,
  declare function = {
      spherical(\alt) = \spradius * cos(\alt);
      equidistance(\alt) = \spradius - \spradius*\alt/90;
      altradius(\alt) = \spprojection(\alt); 
      aziangle(\x) = 90 - \x;
  }
} 

\tikzdeclarecoordinatesystem{sunpath}% ^^A
{
    \tikzset{cs/.cd,azi=0,alt=0,#1}
    \tikzmath{      
      \r = altradius(\tikz@cs@alt);
      \angle = aziangle(\tikz@cs@azi);
    }
    \pgfpointadd{\pgfpointxy{0}{0}}{% ^^A        
        \pgfpointpolarxy{\angle}{\r}
    }
}
\makeatother 

%
\tikzset{
  sunpath grid/.style={help lines,color=blue!45!white!80},
  sunpath tick/.style={draw,thick,color=blue!90!white!80},
  sunpath minor tick/.style={draw,thin,color=blue!90!white!80},
  altitude label/.style={font=\footnotesize\sffamily,fill=white,minimum width={width("90")+2pt},inner sep=0.5pt},
  azimuth label/.style={font=\footnotesize\sffamily,minimum width={width("360")+2pt},inner sep=0.5pt},
}


\NewDocumentCommand\drawgeodirection{}{
  \foreach \dname / \dgrad \ in {N/0, E/90, S/180, W/270}{
      \tikzmath{
        \polarangle = aziangle(\dgrad);
      }
      \coordinate (D) at (\polarangle:\spradius cm + 22pt);
      \node[anchor=270-\dgrad] at (D) {\dname};
  };
}


\NewDocumentCommand\drawaltitudecirle{m}{
  \foreach \altitude in #1 {
    \coordinate (A) at (sunpath cs:azi=0,alt=\altitude) ;  
    \path[draw,sunpath grid] (0,0) circle[radius=altradius(\altitude)];
  }
}


\NewDocumentCommand\drawaltitudelabel{m}{
  \foreach \altitude in #1 {
      \coordinate (A) at (sunpath cs:azi=0,alt=\altitude) ;      
      \node [anchor=east,altitude label] at (A) {\altitude};  
    }
}

\NewDocumentCommand\drawazimuthlabel{m}{
  \foreach \azimuth in #1 {
      \tikzmath{
        \polarangle = aziangle(\azimuth);
      }
      \coordinate (D) at (\polarangle:\spradius cm + 13pt);
      \node[azimuth label] at (D) {\azimuth};
  }
}


\NewDocumentCommand\drawazimuthline{m m m}{
  \foreach \azimuth in #1{
      \draw[sunpath grid] (sunpath cs:azi=\azimuth,alt={#2}) -- (sunpath cs:azi=\azimuth,alt={#3});
  }
}


\NewDocumentCommand\drawazimuthtick{}{
  \foreach \azimuth in {10,20,...,360}{
    \tikzmath{      
        \pa = aziangle(\azimuth);
    }
    \path[sunpath tick] (\pa:\spradius) -- (\pa:{\spradius cm+6pt});
  }
  
  \foreach \azimuth in {1,2,...,360}{
    \tikzmath{      
        \pa = aziangle(\azimuth);
    }
    \path[sunpath minor tick] (\pa:\spradius) -- (\pa:{\spradius cm+2.5pt});
  }
  
  \foreach \azimuth in {15,30,...,360}{
      \tikzmath{      
          \pa = aziangle(\azimuth);
      }
      \path[sunpath minor tick] (\pa:\spradius) -- (\pa:{\spradius cm+5pt});
    }  
}
\endinput



